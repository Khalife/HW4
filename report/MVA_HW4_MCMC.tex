%%%%%%%%%%%%%%%%%%%%%%%%%%%%%%%%%%%%%%%%%
% University Assignment Title Page 
% LaTeX Template
% Version 1.0 (27/12/12)
%
% This template has been downloaded from:
% http://www.LaTeXTemplates.com
%
% Original author:
% WikiBooks (http://en.wikibooks.org/wiki/LaTeX/Title Creation)
%
% License:
% CC BY-NC-SA 3.0 (http://creativecommons.org/licenses/by-nc-sa/3.0/)
% 
% Instructions for using this template:
% This title page is capable of being compiled as is. This is not useful for 
% including it in another document. To do this, you have two options: 
%
% 1) Copy/paste everything between \begin{document} and \end{document} 
% starting at \begin{titlepage} and paste this into another LaTeX file where you 
% want your title page.
% OR
% 2) Remove everything outside the \begin{titlepage} and \end{titlepage} and 
% move this file to the same directory as the LaTeX file you wish to add it to. 
% Then add \input{./title page_1.tex} to your LaTeX file where you want your
% title page.
%
%%%%%%%%%%%%%%%%%%%%%%%%%%%%%%%%%%%%%%%%%

%----------------------------------------------------------------------------------------
%	PACKAGES AND OTHER DOCUMENT CONFIGURATIONS
%----------------------------------------------------------------------------------------

\documentclass[12pt]{article}
\usepackage{graphicx}
\usepackage[utf8]{inputenc}  
\usepackage[T1]{fontenc} 
\usepackage[top=1cm,bottom=1cm,left=0.5cm,right=1.5cm,asymmetric]{geometry}
\usepackage{amsfonts}
\usepackage{graphicx}
\usepackage{algorithm}
\usepackage{algpseudocode}
\usepackage{amsmath}
\usepackage{caption}
\usepackage{subcaption}
\usepackage{float}
\usepackage{subfig}
\usepackage{fancyhdr}
\pagestyle{fancy}
\renewcommand{\footrulewidth}{1pt}
\fancyhead[R]{\textit{Master MVA : Simulation based learning}}
\fancyfoot[L]{\textit{}}
%\usepackage{unicode-math}
%\setmathfont{XITS Math}
%\setmathfont[version=setB,StylisticSet=1]{XITS Math}
\usepackage{array,multirow,makecell}
\setcellgapes{1pt}
\makegapedcells
\newcolumntype{R}[1]{>{\raggedleft\arraybackslash }b{#1}}
\newcolumntype{L}[1]{>{\raggedright\arraybackslash }b{#1}}
\newcolumntype{C}[1]{>{\centering\arraybackslash }b{#1}}

\pagestyle{fancy}
\renewcommand{\footrulewidth}{1pt}
\fancyfoot[L]{\textit{}}
\newcommand{\cond}{(x_i|x_{\pi_i})}

%\usepackage{caption}
%\usepackage{subcaption}


%\usepackage{unicode-math}
%\setmathfont{XITS Math}
%\setmathfont[version=setB,StylisticSet=1]{XITS Math}


%\geometry{hmargin=1.5cm,vmargin=2cm}   

\geometry{hmargin=2.5cm,vmargin=2cm}   
\begin{document}
	
	\section*{Homework 4 (HW4 Part B)}
	\section*{Oussama Ennafii \& Sammy Khalife}
	\subsubsection*{25/02/2015}
	
\section*{Step 1}
\section*{Step 2}	
\section*{Step 3}
1.~\\
~\\
2. Let us factorize the expression of $\tilde{\pi}_{\theta}(u,w)$  under the form :
\begin{eqnarray*}
	\tilde{\pi}_{\theta}(u,w)&=& h(\theta,  w, \beta, \sigma , ...) exp(-\frac{1}{2}u^t u - \sum_{i=1}^{N}\frac{ w_i}{2}\sigma^2(z_i'u)^2 +\sum_{i=1}^{N}\sigma(Y_i-\frac{1}{2})z_i'u- w_{i}x_i'\beta \sigma z_i'u)~\\
	&=& h(\theta,  w, \beta, \sigma , ...)exp(-\frac{1}{2}u^t (I+\sigma^{2}\sum_{i=1}^{n}\frac{ w_i}{2}z_i z_i')u + \sigma<u, \sum_{i=1}^{N}((Y_i-\frac{1}{2})- w_{i}x_i'\beta)z_i>)
	\end{eqnarray*}
Where h does not depend on u. We can directly identify the density of a Gaussian :
$$ \tilde{\pi}_{\theta}(u | w) \quad \propto \quad exp(-\frac{1}{2}u^t (I+\sigma^{2}\sum_{i=1}^{n}\frac{ w_i}{2}z_i z_i')u + \sigma<u, \sum_{i=1}^{N}((Y_i-\frac{1}{2})- w_{i}x_i'\beta)z_i>)$$
i.e
$$ \boxed{\tilde{\pi}_{\theta}(u | w) \quad \propto \mathcal{N}(\mu_{\theta},\Gamma_{\theta}(w))}$$
~\\
With $\boxed{\Gamma_{\theta}(w)=(I+\sigma^{2}\sum_{i=1}^{n}\frac{ w_i}{2}z_i z_i')^{-1}}$, and $\boxed{\mu_{\theta}(w)=\sigma \Gamma_{\theta}(w)\sum_{i=1}^{N}((Y_i-\frac{1}{2})-w_ix_i'\beta)z_i}$.~\\
~\\
3. We use the same reasoning as the previous question. 
\begin{eqnarray*}
	\tilde{\pi}_{\theta}(w | u) & \underset{w}\propto & \prod_{i=1}^{N}\rho(w_i)\textbf{1}_{R^{+}}(w_i)exp(-\frac{w_i}{2}(x_i'\beta+\sigma z_i'u)^2) \\
	& \underset{w}\propto & \prod_{i=1}^{N}Zcosh(\frac{x_i'\beta + \sigma z_i'u}{2})\textbf{1}_{R^{+}}(w_i)exp(-\frac{w_i}{2}(x_i'\beta+\sigma z_i'u)^2)\\
	\end{eqnarray*}~\\
	Since $cosh(x)=cosh(|x|)$,
\begin{eqnarray*}
	\tilde{\pi}_{\theta}(w | u)
	& \underset{w}\propto & \prod_{i=1}^{N}Zcosh(\frac{|x_i'\beta + \sigma z_i'u|}{2})\textbf{1}_{R^{+}}(w_i)exp(-\frac{w_i}{2}(x_i'\beta+\sigma z_i'u)^2)\\
	& \underset{w}\propto & \prod_{i=1}^{N}\bar{\pi}(w_i ;|x_i'\beta + \sigma z_i' u|)
\end{eqnarray*}~\\
Since the last line a function normalized in w, i.e
\begin{eqnarray*}
	\int_{\Omega}\prod_{i=1}^{N}\bar{\pi}(w_i ;|x_i'\beta + \sigma z_i' u|)dw &=& \prod_{i=1}^{N} \int_{\Omega_i}\bar{\pi}(w_i ;|x_i'\beta + \sigma z_i' u|)dw_i\\
	&=& 1
\end{eqnarray*}~\\    	
We have 
$$\boxed{\tilde{\pi}_{\theta}(w | u)= \prod_{i=1}^{N}\bar{\pi}(w_i ;|x_i'\beta + \sigma z_i' u|)}$$
~\\
4.~\\
~\\
5. To sample $(w_{1},...,w_{N},u)$ from $\tilde{\pi}_{\theta}(u,w)$, we will use a Gibbs Sampler 	

\begin{algorithm}
	\caption{Gibbs Sampler to sample from $\tilde{\pi}_{\theta}$}\label{RS}
	Given, $w_{1}^{(0)}, ..., w_{N}^{(0)}$~\\
	~\\
	Loop on k~\\
	Draw $u^{(k+1)}$ from $\pi_{\theta}(. | w^{(k)}) = \mathcal{N}(\mu_{\theta},\Gamma_{\theta}(\omega))$~\\
	-for i=1:N~\\
	$w_{i}^{(k+1)}$ drawn thanks to HW1Sampler(.,$|x_i'\beta + \sigma z_i' u^{(k+1)}|/2)$\\
	-end for i~\\
	end k
\end{algorithm}~\\
~\\
6.~\\
~\\	

	
	
\end{document}